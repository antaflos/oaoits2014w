\documentclass[a4paper,11pt,onecolumn]{scrartcl}

\usepackage{etex}
\usepackage{palatino}
\usepackage[english]{babel}
\usepackage[T1]{fontenc}
\usepackage[utf8]{inputenc}
\usepackage[tracking]{microtype}
\usepackage{url}
\usepackage{doi}
\usepackage{hyperref}
\usepackage[]{color}
\usepackage{siunitx}
\usepackage[shortlabels]{enumitem}
\usepackage{graphicx}

\graphicspath{{images/}}

\definecolor{grey}{rgb}{0.5,0.5,0.5}
\definecolor{darkgreen}{cmyk}{0.7, 0, 1, 0.5}
\definecolor{LinkColor}{rgb}{0,0,0.5}

\setlist{noitemsep}

\pagestyle{headings}
\hypersetup{%%Hyperref stuff
	pageanchor=false,
	pdfpagelabels=false,
	final,
	colorlinks=true, 
	linkcolor=LinkColor,
	citecolor=LinkColor,
	filecolor=LinkColor,
	menucolor=LinkColor,
	urlcolor=LinkColor,
	pdftitle={}
}

\newcommand{\command}[1]{`\texttt{#1}'}
\newcommand{\code}[1]{\small\texttt{#1}\normalsize}

\begin{document}

\title{Business Continuity Management with focus on Backup and Restore}
\publishers{\small{Organizational Aspects of IT Security\\VU 2.0 188.312 2014W}}
\subject{Assignment 1}

\author{Andreas Ntaflos\thanks{Matrikelnummer e0326302, \href{mailto:daff@pseudoterminal.org}{\code{daff@pseudoterminal.org}}}
\and Andreas Ntaflos\thanks{Matrikelnummer e1328682, \href{mailto:e1328682@student.tuwien.ac.at}{\code{e1328682@student.tuwien.ac.at}}}
\and Christoph Seidl\thanks{Matrikelnummer e0427434, \href{mailto:e0427434@student.tuwien.ac.at}{\code{e0427434@student.tuwien.ac.at}}}}

\maketitle

\begin{abstract}
In this document we examine the implementation of company-wide data
backup and restore solutions from a Business Continuity Management
approach, specifically in a Disaster Recovery context. We discuss
relevant concepts such as Business Continuity Planning and Disaster
Recovery Planning, key metrics such as Recovery Point Objective,
Recovery Time Objective and others which define the requirements of
backup and restore solutions and techniques, and how these metrics must
be based on a Business Impact Analysis to be useful in the business
context. We look at the risks associated with backup and restore from an
information security point of view and how to control for these risks
when implementing a backup solution.

After reviewing different, representative backup solutions and
techniques and discussing their merits and drawbacks---again from an
information security point of view---we highlight various challenges and
potential problems a company faces when implementing backup solutions.
Finally we discuss future developments and how backup and disaster
recovery requirements may change over time.
\end{abstract}

\section{Business Continuity Management}
% What is BCM?
% Business Continuity Plan, Disaster Recovery Plan
% MTD, RPO, RTO, WRT

\emph{Business Continuity Management} (BCM) is a strategic management
process covering \emph{Business Continuity Planning} (BCP) and
\emph{Disaster Recovery Planning} (DRP) to ensure critical business
functions can continue during and after major incidents (flood, fire,
earthquake, terrorism, vandalism, \ldots) with minimal
loss of operations, systems and life.

BCM provides a framework for integrating resilience with the capability
for effective responses that protects the interests of an organisation's
key stakeholders. The main objective of BCM is to allow the organisation
to continue to perform business operations under various
conditions~\cite{harriscissp}.

In BCM the two most important artifacts are the \emph{Business
	Continuity Plan} (BCP) and the \emph{Disaster Recovery Plan} (DRP).
The BCP ensures that critical business functions can continue to operate
during a disaster and its aftermath. This may include moving critical
equipment to standby locations, bringing up emergency power generators,
performing normally automated business operations manually or
temporarily hiring additional personnel.

The DRP is usually a part of the BCP and is focused on the IT
infrastructure itself. It defines how to prepare the IT infrastructure
for a disaster, what to do in the event of a disaster and how to restore
IT operations so that the business can go back to operating normally.
The DRP's scope is thus much narrower and technology-focused than the
BCP as a whole.

Disaster recovery can be quantified by four key metrics:

\begin{description}
	\item[Maximum Tolerable Downtime (MTD)] The maximum outage
		time of a critical business function or system that can be
		endured before the impact on the company becomes unacceptably
		severe
	\item[Recovery Point Objective (RPO)] The amount of data loss,
		measured in time, that can be sustained from a disaster without
		causing irreparable damage to the company and its business
		functions
	\item[Recovery Time Objective (RTO)] The time it may take to restore
		business functions after a disaster
	\item[Work Recovery Time (WRT)] The time it may take to recover data
		and bring infrastructure back so that normal operation can
		continue
\end{description}

The MTD consists of the RTO and the WRT while the RPO exists
independently of the MTD. Figure~\ref{fig:mtd_all} shows the metrics in
relation to each other. The MTD is also known as the \emph{Maximum
	Tolerable Period of Disruption} (MTPD).

It should be noted that IT vendors of backup and DR solutions tend to
conflate RTO and WRT into just RTO, which makes sense considering that a
business function or process can hardly be declared ``restored'' when it
is still missing its key business data. We will refer to RTO in the same
manner in the remainder of this document.

\begin{figure}[t]
	\begin{center}
		\begin{scriptsize}
			\def\svgwidth{\textwidth}
			\input{images/mtd_05_all.pdf_tex}
		\end{scriptsize}
	\end{center}
	\caption{The disaster recovery metrics in relation to each other on
		a time line}
	\label{fig:mtd_all}
\end{figure}

The values for MTD, RPO and RTO usually differ between systems,
business functions and processes. It is thus important not to define
them ad-hoc and on a whim but on a solid understanding of the business
functions in question. This understanding comes from the Business Impact
Analysis.

\section{Business Impact Analysis}
% What is it? What information does it provide?
% How to conduct?

The \emph{Business Impact Analysis} (BIA) is a functional analysis that
identifies the various business processes, functions, activities,
resources and systems that define the business. It assigns them
criticality levels and identifies threats and vulnerabilities for these
functions and calculates the risk for each. The Business Impact Analysis
is a key step in creating a disaster recovery plan and defining backup
requirements.

In any reasonably sized organisation a BIA cannot be conducted by a
single person or team without the help and input of key employees such
as department heads and process owners. Thus the data for the BIA is
gathered by means of interviews, questionnaires, workshops, existing
documentation, etc. The result of the data gathering phase is a
comprehensive list of business processes, functions and resources
relevant to the organisation.

After all business processes and functions have been identified these
resources need to be categorised by their criticality to the
organisation. A resource is regarded as critical when its availability
is imperative for the survival of the organisation and downtime of that
resource is unacceptable. The most critical resources will receive the
highest priority when recovering from a disaster while less- or
non-critical systems will receive attention only after all critical
systems have been dealt with. It is at this step that concrete values
for MTD (MTPD), and later RPO and RTO of a critical resource can first
be estimated.

MTD values can be categorised into \emph{Nonessential}, \emph{Normal},
\emph{Important}, \emph{Urgent} and \emph{Critical}. The identified
business resources are each assigned one of these categories and treated
accordingly. The shorter the MTD, the more critical the resource, and
the higher the priority it will be afforded when recovering.

When conducting the BIA it is important to consider interdependencies
between business functions, resources or systems. Most business
functions do not stand alone but depend on other business functions and
systems, so when estimating the MTD these dependencies need to be taken
into account. 

The next step is threat and risk assessment. This includes identifying
vulnerabilities, threats and to the most critical business resources,
determining the impact each threat may have and the likelihood of its
occurrence. Vulnerabilities in resources are single points of failure or
security issues, threats may include sabotage, vandalism or theft but
also natural disasters or major utilities outages. The risk to a
business function or resource is thus calculated as $\textrm{Risk} =
\textrm{Threat} \times \textrm{Impact} \times \textrm{Probability}$. 

At this point it is important to note that while identifying and
planning for specific threats with high impact or likelihood is prudent
it is not useful to try to account and prepare for \emph{all} possible
threats. Instead the organisation should prepare for the loss of any or
all business resources, regardless of specific threats. This will
provide the organisation with the proper flexibility when the time of
disaster recovery comes.

After determining the MTD for critical business functions taking into
account their dependencies on other business functions and resources
reasonable values for RPO and RTO can be defined. These are more
meaningful than just the MTD in a disaster recovery context and are an
important basis for each critical business function's disaster recovery
strategy.

\section{Disaster Recovery Strategies}
% Backup/Restore and DR overlap greatly and cannot be discussed
% independently of each other.
% Requirements for backup systems and implementations based on RPO and RTO/WRT
% Point in time recovery? 
% Disaster recovery details and strategies: failover, failback; hot, warm, cold standby
% Risks: availability, integrity, confidentiality
% Risks in availability: locality of backups (where are they stored?)
% Risks in integrity: 
% Controls: restore tests, integrity tests, backup monitoring
% Archiving:
% Financial and business records need to be archived for seven years or longer
% § 132 Bundesabgabenordnung

\section{Review of backup tools and solutions}

\section{Challenges and potential problems}

\section{Future developments and requirements}

\nocite{*}
\bibliographystyle{plain}
\bibliography{backuprestore}

\end{document}
